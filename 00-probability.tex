\documentclass[11pt]{article}
\usepackage{fullpage}
\usepackage{amsthm}

\usepackage{amsthm,amsmath,amsfonts,amssymb,amstext,enumitem}
\usepackage{latexsym,ifthen,url,rotating}
%\usepackage{color}

\usepackage[usenames,dvipsnames]{color}

% --- -----------------------------------------------------------------
% --- Document-specific definitions.
% --- -----------------------------------------------------------------

\newcommand{\todo}[1]{{\color{red}[TODO:{#1}]}}

\newtheorem{problem}{Problem}
\newtheorem{fact}{Fact}
\newtheorem{exercise}{Exercise}
\newtheorem{theorem}{Theorem}
\newtheorem{definition}{Definition}
\newtheorem{notation}{Notation}
\newtheorem{lemma}{Lemma}
\newtheorem{example}{Example}

\newcommand{\getsr}
  {{\:\stackrel{\raisebox{-2pt}{${\scriptscriptstyle \hspace{0.2em}\$}$}}
   {\leftarrow}\:}}
\newcommand{\points}[1]{\textbf{({#1} pts)}}

\newcommand{\Colon}{\ : \ }
\newcommand{\st}{\mathsf{state}}
\newcommand{\msgsp}{\mathcal{M}}
\newcommand{\keys}{\mathcal{K}}
\newcommand{\kg}{\mathcal{K}}
\newcommand{\enc}{\mathcal{E}}
\newcommand{\decc}{\mathcal{D}}
\newcommand{\MAC}{\mathrm{MAC}}
\newcommand{\RMAC}{\mathrm{RMAC}}

\newcommand{\pk}{pk}
\newcommand{\sk}{sk}

\newcommand{\AES}{\mathsf{AES}}

\newcommand{\algorithm}[1]{\textbf{Alg} {#1}}

\newcommand{\calO}{\mathcal{O}}

\newcommand{\dlog}{\mathrm{dlog}}

\newcommand{\Adv}{\mathbf{Adv}}
\newcommand{\AdvPRF}[2]{\Adv^{\mathrm{prf}}_{#1}({#2})}
\newcommand{\AdvCPA}[2]{\Adv^{\mathrm{ind{-}cpa}}_{#1}({#2})}
\newcommand{\AdvCCA}[2]{\Adv^{\mathrm{ind{-}cca}}_{#1}({#2})}
\newcommand{\AdvKR}[2]{\Adv^{\mathrm{kr}}_{#1}({#2})}
\newcommand{\AdvCKR}[2]{\Adv^{\mathrm{ckr}}_{#1}({#2})}
\newcommand{\AdvRMR}[2]{\Adv^{\mathrm{rmr}}_{#1}({#2})}
\newcommand{\AdvCR}[2]{\Adv^{\mathrm{cr}}_{#1}({#2})}
\newcommand{\AdvUFCMA}[2]{\Adv^{\textrm{uf{-}cma}}_{#1}({#2})}
\newcommand{\AdvDL}[2]{\Adv^{\mathrm{dl}}_{#1}({#2})}

\newcommand{\Exp}{\mathbf{Exp}}
\newcommand{\ExpOW}[1]{\Exp^{\mathrm{ow}}({#1})}
\newcommand{\ExpCKR}[2]{\Exp^{\mathrm{ckr}}_{#1}({#2})}
\newcommand{\ExpRMR}[2]{\Exp^{\mathrm{rmr}}_{#1}({#2})}

\newcommand{\concat}{{\,\|\,}}
\newcommand{\xor}{\oplus}
\newcommand{\bits}{\{0,1\}}

\newcommand{\tcolh}{T^{\mathrm{col}}_h}
\newcommand{\tcolH}{T^{\mathrm{col}}_{H^2}}
\newcommand{\Hcomb}{H^{1\|2}}
\newcommand{\Hxor}{H^{1\oplus2}}

\newcommand{\EXP}{\textrm{EXP}}
\newcommand{\MODEXP}{\textrm{MOD{-}EXP}}
\newcommand{\ADD}{\textrm{ADD}}
\newcommand{\MULTIMODEXP}{\textrm{MULTI{-}MOD{-}EXP}}
\newcommand{\MUL}{\textrm{MUL}}
\newcommand{\MOD}{\textrm{MOD}}

\newcommand{\GG}{\mathbb{G}}
\newcommand{\ZZ}{\mathbb{Z}}

\newcommand{\rvrange}{\mathcal{R}}
\newcommand{\rspace}{\mathcal{C}}


% --- -----------------------------------------------------------------
% --- Lecture notes formatting macros
% --- -----------------------------------------------------------------

%
% The following commands set up the lecnum (lecture number)
% counter and make various numbering schemes work relative
% to the lecture number.
%
\newcounter{lecnum}
%\renewcommand{\thepage}{\thelecnum-\arabic{page}}
\renewcommand{\thesection}{\thelecnum.\arabic{section}}
\renewcommand{\theexercise}{\thelecnum.\arabic{exercise}}
\renewcommand{\theexample}{\thelecnum.\arabic{example}}
\renewcommand{\thedefinition}{\thelecnum.\arabic{definition}}
\renewcommand{\theequation}{\thelecnum.\arabic{equation}}
\renewcommand{\thefigure}{\thelecnum.\arabic{figure}}
\renewcommand{\thefact}{\thelecnum.\arabic{fact}}
\renewcommand{\thetable}{\thelecnum.\arabic{table}}


%
% The following macro is used to generate the header.
%
\newcommand{\lecture}[2]{
   %\pagestyle{myheadings}
   %\thispagestyle{plain}
   \newpage
   \setcounter{lecnum}{#1}
   \setcounter{page}{1}
   \noindent
   \begin{center}
   \framebox{
      \vbox{\vspace{2mm}
    \hbox to 6.28in { {\bf CMSC 28400 Introduction to Cryptography
                        \hfill Autumn 2019} }
       \vspace{4mm}
       \hbox to 6.28in { {\Large \hfill #2 \hfill} }
       \vspace{2mm}
       \hbox to 6.28in { {\it Instructor: David Cash} \hfill }
      \vspace{2mm}}
   }
   \end{center}
   %\markboth{Lecture #1: #2}{Lecture #1: #2}
   \vspace*{4mm}
}





% --- -----------------------------------------------------------------
% --- The document starts here.
% --- -----------------------------------------------------------------
\begin{document}
%\lecture{**LECTURE-NUMBER**}{**DATE**}{**LECTURER**}{**SCRIBE**}
\lecture{0}{Notes \#0: Discrete Probability Background}

\tableofcontents

\noindent\hrulefill
\bigskip

These notes review some essential facts about probability theory and
modular arithmetic that we'll use routinely.

%\section{Discrete Probability Theory and Randomized Algorithms}

As we will quickly see, \emph{probability theory} is central to cryptography:
When we want to pick a key that our adversaries do not ``know'', simply choosing
a key at random is the best way to do so meaningfully. Our treatment
of probability in CMSC 28400 will not typically be so formal, but I find it
very useful to have a precise foundation (i.e. formal definitions and
basic theorems) to refer to when things get complicated.

\section{Distributions and Probability Measures}

The type of probability we review now is \emph{discrete} in the sense
that all of the sets involved are countable.
(By \emph{countable} we mean \emph{finite or countably infinite}.)

\subsection{Probability Distributions}
Let's start with a definition.
\begin{definition}
    Let $\Omega$ be a non-empty countable set. A \emph{probability mass
    function (pmf)} or \emph{distribution on $\Omega$} is a function $p:\Omega \to [0,1]$ such that
    $\sum_{w\in\Omega}p(w)=1$.
\end{definition}
A distribution $p$ can be seen as a way of assigning to every element
of $\Omega$ a number between $0$ and $1$ (a ``probability'') so that the
probabilities sum to one.  When modeling the outcome a fair coin, we could
take $\Omega = \{0,1\}$ (representing Heads and Tails as we like)
and let $p(0)=p(1)=1/2$. To model a biased coin,
we could take the same $\Omega$ but with $p(0) = 1/4, p(1) = 3/4$.
For rolling a die, we can take $\Omega = \{1,2,3,4,5,6\}$, and so on.
We are free to choose $\Omega$ to model our problem.

Note that when $\Omega$ is infinite, the sum is in fact a limit. Since $\Omega$
is countable, the order of elements in the sum does not matter.   We caution
that when $\Omega$ is uncountable, dramatically more complicated definitions
are required to give a useful theory. We will not stray into such territory for
CMSC 28400, and even the formal details of countable sums will not be important.

\begin{definition}
    Let $p$ be a distribution on $\Omega$. We say that $p$ is
    \emph{uniform} if $p(w) = p(w')$ for all $w,w'\in\Omega$.  When
    $\Omega$ is finite, this implies $p(w) = \frac{1}{|\Omega|}$.
\end{definition}

\begin{example}
    If $\Omega$ is infinite, then there does not exist a uniform distribution on
    $\Omega$. To prove this, take some $w\in\Omega$ (recall that $\Omega$ is
    non-empty). Then either $p(w)=0$ or $p(w)\neq 0$. If $p(w)=0$,
    then $\sum_{w\in\Omega} p(w) = 0$ since $p$ is uniform. If on the other
    hand
    $p(w)=c>0$, then $\sum_{w\in\Omega} p(w) =\sum_{w\in\Omega}
    c\rightarrow \infty$ because $p$ is uniform and $\Omega$ is infinite.
    Either way, $\sum_{w\in\Omega}p(w) \neq 1$ and $p$ is not a distribution on
    $\Omega$. 
\end{example}
Consider this point of view on uniform probability distributions: If you pick
an element of $\Omega$ according the uniform distribution without showing me,
then I effectively have ``no idea'' what you picked.  From this perspective,
the latter part of the example gives a deep fact: When $\Omega$ is countably 
infinite, it's impossible to pick a
sample from $\Omega$ so that I have ``no idea'' what you picked, because it's
impossible to pick a \emph{uniform} sample! Even more remarkably, it \emph{is}
possible to pick a uniform sample from an \emph{uncountable} set, a fact that
can be explained once the theory of measures is developed.

\subsection{Probability Measures}

The theory of discrete probability could, in principle, begin and end with
distributions only. But things get more interesting when we introduce other
perspectives on understanding distributions. The first such perspective is
\emph{probability measures}, which shift from looking at the probability of
individual elements $w\in\Omega$ to the probability of \emph{subsets of
$\Omega$}.  Defining ``the probability of a subset'' isn't quite a simple as
distributions, which define ``the probability of a specific outcome''.


In this following definition, we write $2^\Omega$ to be the collection of
all subsets of $\Omega$. When $\Omega$ is finite, $|2^\Omega| = 2^{|\Omega|}$.
\begin{definition}
    Let $\Omega$ be a non-empty countable set. We say that a function
    $\Pr:2^\Omega \to [0,1]$ is a \emph{discrete probability measure on
    $\Omega$} if the following hold:
    \begin{itemize}
        \item $\Pr[\Omega] = 1$.
        \item For any countable sequence 
            $E_1,E_2,\ldots$ of disjoint subsets of $\Omega$,
            $\Pr[\bigcup_{i=1}^\infty E_i] = \sum_{i=1}^\infty\Pr[E_i]$.
            This property of $\Pr$ is called \emph{countable additivity}.
    \end{itemize}
    When $\Pr$ is a probability measure on $\Omega$, 
    the pair $(\Omega,\Pr)$ is called a \emph{discrete probability space}.
    In these notes we will just call it a \emph{probability space}.
\end{definition}
Why should this be the definition of a probability measure? It's not because
it's the most intuitive definition of what probability should be. As far as
I can tell, this definition is used because it is very compact (just two
rules!), and it implies that $\Pr$ has \emph{all} of structure that
corresponds to anything you'd intuitively expect probability to
satisfy.\footnote{ As usual, I caution that when $\Omega$ is uncountable,
another, more complicated, definition is required because often there won't
exist a measure satisfying this definition. If you read another mathematical
text on probability, they usually refer to this more complicated
definition. For CMSC 28400, don't worry about
this case.}

Note that the additivity condition includes finite 
sequences $E_1,\ldots,E_n$ of disjoint sets; We can take $E_j=\emptyset$
for all $j>n$, which will technically be a sequence of disjoint sets.

\begin{definition}
    Let $\Omega$ be a non-empty countable set and $p$ be a distribution on
    $\Omega$. 
    The function $\Pr:2^\Omega \to [0,1]$ defined by
    \[
        \Pr[E] = \sum_{w\in E}p(w)
    \]
    is called \emph{the probability measure (on $\Omega$) induced by $p$}.
\end{definition}
\begin{exercise}
    Verify that $\Pr$ is indeed a probability measure according to the
    definition.
\end{exercise}
Note that $\Pr$ is a function that takes as input a subset $E$ of $\Omega$,
and outputs a number between $0$ and $1$. Instead of writing $\Pr(E)$ we write
$\Pr[E]$, but this has exactly the same meaning. We use the $\Pr[E]$ notation
to help us keep straight what is a ``probability''.   
It can be easily shown that the sums defining $\Pr$ all converge.
Finally, note that
when $E=\emptyset$, the sum defining $\Pr[E]$ is trivial and taken to be zero.


The dependence on the distribution $p$ in the notation $\Pr[E]$ is implicit.
Note that $\Pr$ depends on $p$; If we were ever in a setting with multiple
distributions, we would need different notation like $\Pr_p$ to keep this
dependence straight. Thankfully we will not to need to do so, but it is
important to internalize that $\Pr$ is a \emph{specific} function on subsets,
defined by context, and \emph{not a universally meaningful symbol} like, say
$d/dx$ from calculus. I have noticed that this custom is often fundamentally
confusing for people encountering formal probability theory for the first time.

\begin{exercise}
    Fix some non-empty $\Omega$, and suppose you are given a function
    $\Pr:2^{\Omega}\to[0,1]$ and told that $\Pr$ is induced by some
    distribution $p$. Show that $\Pr$ determines $p$. Conclude that there
    is a one-to-one correspondence between distributions and the probability
    measures they induce.
\end{exercise}
When $p$ is uniform, we also say that the measure $\Pr$ induced by $p$ is
uniform. 

\begin{exercise}
    Show that
    when $\Omega$ is finite and $\Pr$ is uniform, 
    we have that
    \[
        \Pr[E] = \frac{|E|}{|\Omega|}.
    \]
\end{exercise}
Thus for uniform measures, calculating probabilities reduces to calculating the
sizes of $E$ and $\Omega$.

\begin{example}
Equating the notion of an event with a subset of $\Omega$ gives us a
convenient language for connecting intuitive descriptions of outcomes
with the formalism. For instance, let $\Omega = \bits^n$ with the uniform
distribution.
    $E = \{0\|x \ : \ x\in\bits^{n-1}\}$. Then 
    \[
        \Pr[\text{``a uniformly random $n$-bit string starts with zero''}] = \Pr[E] = \frac{1}{2}.
    \]
    Another example takes 
    $F = \{0^{n/2}\|x \ : \ x\in\bits^{n/2}\}$. Then 
    \[
        \Pr[\text{``a uniformly random $n$-bit string starts with $n/2$ zeros''}] =
        \Pr[F] = \frac{2^{n/2}}{2^n} = \frac{1}{2^{n/2}}.
    \]
    With calculations like this one usually skips the formalism and jumps
    straight to the answer; In a sense the theory is justified by giving the
    right answer more than the other way around.  But it is good to know what
    exactly is being formalized, in case a proof
    makes a more subtle jump.
\end{example}

The following exercise begins to justify the definition of a probability
space; From those simple conditions, a lot of intuitively-true properties
must also hold.
\begin{exercise}
    Let $(\Omega,\Pr)$ be a probability space.
    Prove the following: 
    \begin{itemize}
        \item $\Pr[\Omega] = 1$
        \item For disjoint events $E,F\subseteq \Omega$, 
            $\Pr[E\cup F] = \Pr[E] \cup \Pr[F]$.
        \item For any two events $E,F\subseteq \Omega$, 
            $\Pr[E\cup F] = \Pr[E] \cup \Pr[F] - \Pr[E\cap F]$.
        \item For any two events $E,F\subseteq \Omega$, if $E\subseteq F$
            then $\Pr[E] \leq \Pr[F]$.
        \item For an event $E$, let $E^c = \Omega \setminus E$ be the
            compliment of $E$ (i.e. everything in $\Omega$ that is not in
            $E$). Then $\Pr[E^c] = 1 - \Pr[E]$.
    \end{itemize}
\end{exercise}
All of the above facts are true for infinite $\Omega$, but rigorous proofs
require some appeal to calculus.

One could go on generating lists of theorems like in the exercises. The
essential idea if the following: Any formula relating the \emph{size} of
events as sets remains true when we replace with ``$|E|$'' with ``$\Pr[E]$''
everywhere, up to some corner cases that occur when some elements of $\Omega$ have
probability zero. So for example: $|E\cup F| \leq |E| + |F|$, and
thus $\Pr[E\cup F] \leq \Pr[E] + \Pr[F]$. But it might be that $\Pr[E]=0$
yet $|E| > 0$. 

We'll use the following facts later on in the course.

\begin{fact}[Union Bound]
    Let $(\Omega,\Pr)$ be a probability space 
    and let $E_1,\ldots,E_n\subseteq \Omega$ be events. Then
    \[
        \Pr[E_1 \cup \cdots \cup E_n] \leq
        \Pr[E_1] + \cdots + \Pr[E_n].
    \]
\end{fact}

\begin{fact}[Law of Total Probability]
    Let $(\Omega,\Pr)$ be a probability space 
    and let $E,F\subseteq \Omega$ be events. Then
    \[
        \Pr[E] = \Pr[E\cap F] + \Pr[E \cap F^c].
    \]
\end{fact}

\section{Conditional Probability}

Probability starts to get really interesting when you introduction
\emph{conditioning}. This brief introduction probably won't be enough
to give you full intuition for conditional probability, so I recommend
reading up in \cite{?} if you feel rusty.

\begin{definition}
    Let $(\Omega,\Pr)$ be a probability space 
    and let $E,F$ be events with $\Pr[F] \neq 0$. We define
    the \emph{conditional probability of $E$ given $F$} to be
    \[
        \Pr[E|F] = \frac{\Pr[E\cap F]}{\Pr[F]}.
    \]
\end{definition}


One application, which will become clear later, is the following.
\begin{fact}[Lazy Cryptographer's Law of Total Probability]
    Let $(\Omega,\Pr)$ be a probability space and let $E,F$ be events. Then
    \[
        \Pr[E] \leq \Pr[F] + \Pr[E|F^c].
    \]
\end{fact}

\begin{proof}
    By the Law of Total Probability and the Chain Rule,
    \begin{align*}
        \Pr[E] &= \Pr[E\cap F] + \Pr[E \cap F^c] \\
        & \leq \Pr[F] + \Pr[E \cap F^c] \\
        & = \Pr[F] + \Pr[F^c]\Pr[E|F^c] \\
        & \leq \Pr[F] + \Pr[E|F^c].
    \end{align*}
    The first inequality uses $\Pr[E\cap F] \leq \Pr[F]$. The
    second inequality uses the simple fact that $\Pr[F^c] \leq 1$.
\end{proof}
Note this is mixing probabilities of two different ``types'' (conditional and
unconditioned), which is not normally recommended. But anyway this inequality
is used when one wants to bound the probability of $E$ when the probability
that $E$, given $F^c$ is easy to a analyze and the probability of $F$ is easy
to analyze. We will point to examples later in the course, so for now you can
treat it as an exercise to understand the proof.

\begin{exercise}
    Let $\Omega=\{00,01,10,11\}$ and $\Pr$ be the uniform measure on $\Omega$,
    which models choosing two random bits.
    Let $E$ be the event that the  first bit is zero, and $F$ be the
    event that the chosen bits are the same. Verify that $E$ and $F$ are
    independent.
\end{exercise}


\begin{exercise}
    Let $(\Omega,\Pr)$ be a probability space, and let $F$ be an event
    with non-zero probability. Show that the function $\Pr_F:2^\Omega\to[0,1]$,
    defined by $\Pr_F[E] = \Pr[E|F]$ is a probability measure. If $p$
    is the distribution that induces $\Pr$, what distribution induces
    $\Pr_E$?
\end{exercise}


\section{Random Variables}

We next review \emph{random variables}, which are an abstraction to make sense
of informal statements like ``Let $X$ and $Y$ be the outcomes of two fair die
rolls.'' By augmenting our theory with a bit more abstraction, we can increase
the expressiveness and comprehensibility of the theory of probability similar
to how measures (and events) were more powerful and convenient than
distributions.

\begin{definition}
    Let $(\Omega,\Pr)$ be a probability space. A random variable on $(\Omega,\Pr)$ with range $\rvrange$ is a function $X:\Omega\to\rvrange$.
\end{definition}
At first glance this is not a very enlightening definition. Let us
start with some examples.
\iffalse
\begin{example}
    Let $\Omega=\bits^n$ and $\Pr$ but the uniform measure. Define
    $X_i:\Omega \to \bits$ by setting $X_i(w)$ to 
    the $i$-th bit of $w$, and define $Y:\Omega\to\{0,1,\ldots,n\}$
    by setting $Y(w)$ to be the number of $1$ bits of $w$.
    Then $X_1,\ldots,X_n,Y$ are all random variables.
    We have for all $w\in\Omega$,
    \[
        Y(w) = \sum_{i=1}^n X_i(w).
    \]
    One typically expresses this relationship by writing 
    $Y = \sum_{i=1}^n X_i$, leaving out the $w$ entirely. One sees this
    type of notation occasionally in calculus, where you might write 
    $f = g+h$ instead of $f(x) = g(x) + h(x)$.
\end{example}
\fi
\begin{example}
    Let $\Omega=\{00,01,10,11\}$ and $\Pr$ but the uniform measure on $\Omega$. 
    Define $X_1:\Omega \to \bits$ and $X_2:\Omega \to \bits$
    by setting $X_1(w)$ to the first bit of $w$ and $X_2(w)$ to the second
    bit of $w$,
    and define $Y:\Omega\to\{0,1,\ldots,n\}$
    by setting $Y(w)$ to be the number of $1$ bits of $w$.

    Then $X_1,\ldots,X_n,Y$ are all random variables.
    We have for all $w\in\Omega$,
    \[
        Y(w) = X_1(w) + X_2(w)
    \]
    One typically expresses this relationship by writing 
    $Y = X_1+X_2$, leaving out the $w$ entirely. One sees this
    type of notation occasionally in calculus, where you might write 
    $f = g+h$ instead of $f(x) = g(x) + h(x)$.
\end{example}

In this example, we can think of the random variables as \emph{measurements} on
the outcomes in $\Omega$. Above, we can \emph{think} of $Y$ as representing
the outcome of picking a random bit string and then counting the number of
$1$ bits. Of course, when pressed, we must admit that formally $Y$ is a
\emph{function} and not an actual random outcome.

Why should we formalize random variables as functions? The answer will
hopefully be clear after we develop some more concepts using random variables.
But a first benefit is they give us some language for discussing events
compactly, via the following notation.

\begin{notation}
    Let $(\Omega,\Pr)$ be a probability space and let $X:\Omega\to\rvrange$ be
    a random variable on this space.  For $i\in\rvrange$, we define
    \[
        \Pr[X=i] = \Pr[\{w\in\Omega\ : \ X(w)=i\}].
    \]
    Note that $\Pr,X$ are still functions; Prior to this definition,
    the right-hand side of the equation would not make sense.
    The left-hand side, however,  did already make sense: $\Pr$ is a function
    that takes as input subsets of $\Omega$, and $\{w\in\Omega\ : \ X(w)=i\}$
    is such a set.
\end{notation}
The point of this notation is that ``$\Pr[X=i]$'' is a natural notion
to think about: It \emph{should} be the probability that a random variable
takes the value $i$. The notation makes this natural notion precise.
Note that this notation is not $\Pr[X(w)=i]$ -- It omits the $w$, being
consistent with convention mentioned in the previous example.
We remark that this notation is part of why we prefer $\Pr[X=i]$ over
$\Pr(X=i)$: $\Pr$ is a function, but one we use strangely. (This preference
is not universal, particularly not amongst mathematicians).

\begin{example}\label{ex:dicesum}
    Let $\Omega = \{(a,b)\ : \ 1\leq a,b, \leq 6\}$ with the uniform
    probability measure (i.e. we model the outcome of rolling a pair of
    fair dice). Define $X$ on this space as $X(a,b) = a+b$.
    Then
    \begin{align*}
        \Pr[X=4]  
        & = \Pr[\{(a,b)\in\Omega\ : \ X(a,b)=4] \\
        & = \Pr[\{(a,b)\in\Omega\ : \ a+b=4] \\
        & = \Pr[\{(1,3),(2,2),(3,1)\}] \\
        & = 3/36 = 1/12.
    \end{align*}
\end{example}


\begin{definition}
    Let $(\Omega,\Pr)$ be a probability space and let $X:\Omega\to\rvrange$
    be a random variable. The \emph{distribution of $X$}, denoted
    $p_X$, is defined to be
    \begin{align*}
        p_X: \rvrange & \to  [0,1] \\
        i & \mapsto  \Pr[X=i]
    \end{align*}.
\end{definition}

\begin{example}
    Let $X$ be the sum of two fair dice (as in Example~\ref{ex:dicesum}).
    Then $\rvrange = \{2,\ldots,12\}$, and $p_X(2)=1/36, p_X(3)=2/36$, etc.
\end{example}

The next example deserves special attention.
\begin{example}
    Again
    using the same probability space as in Example~\ref{ex:dicesum}.
    Define $Z_1$ to be the outcome of the first roll, and $Z_2$ to be
    the outcome of the second roll. (Formally: $Z_1(a,b) = a$ and
    $Z_2(a,b)=b$.)
    Then $Z_1$ and $Z_2$ have the same distribution:
    That is, $p_{Z_1}$ and $p_{Z_2}$ are \emph{exactly the same function}.
    Both of them map every element of $\{1,2,3,4,5,6\}$ to $1/6$.

    When two (or more) random variables have the same distribution, we say
    they are \emph{identically distributed}.
\end{example}
This example begins to point to the power of random variables: $Z_1$ and $Z_2$
have the same distribution (``uniform on numbers between $1$ and $6$''),
but they are still \emph{different} random variables. 
Intuitively, this is
because they are measuring different dice. Concretely, $Z_1$ and $Z_2$ are
just different functions from $\Omega$ to $\rvrange$.

\begin{definition}
    Let $(\Omega,\Pr)$ be a probability space, and let $X$ and $Y$ be
    random variables on this space with the same range $\rvrange$. 
    We say that $X$ and $Y$ are independent
    if, for every $i,j\in\rvrange$
    \[
        \Pr[X=i,Y=j] = \Pr[X=i]\Pr[Y=j].
    \]
    The notation ``$X=i,Y=j$'' means $X=i$ and $Y=j$ simultaneously; That is,
    \[
        \Pr[X=i,Y=j] = \Pr[\{w\in\Omega \ : \ X(w)=i \text{ and } Y(w)=j\}].
    \]
\end{definition}

\begin{example}
    One can check that $Z_1$ and $Z_2$ from the previous exercise are
    independent, but $Z_1$ and $Z_1+Z_2$ are not.
\end{example}

There is quite a bit more to say about random variables; From here one would
usually learn about \emph{expectation} and \emph{variance}. However CMSC
28400 will not need these concepts.


\section{Algorithms and Randomized Algorithms}

For this class, you won't need to know what a formal \emph{algorithm} is
exactly. But in case you've seen the concept of a Turing Machine or 
(uniform) circuit family, that's what we mean. If you haven't seen those, or
don't recall the definitions, you can think of an \emph{algorithm} as
a piece of code that accepts an input, performs some computation than
can be counted in discrete steps while consuming some amount of memory,
and finally emits an output. 

We will at various times consider algorithms that make internal random choices
as part of their computation. You can think of these as machines with a special
input that should be as many random bits as they need to run. This is akin to
reading from the special file \texttt{/dev/random} in Unix-like operating
systems. For cryptography, we'll be interested in making our algorithms (like
selecting a key) randomized in order to achieve certain security goals.  We'll
also be interested in analyzing the possibility of adversaries using randomized
algorithms, just in case they might help break our systems.

\begin{definition}
    A \emph{randomized algorithm $A$} is an algorithm with a distinguished
    input from some associated finite probability space $\Omega_A$ with the
    uniform measure.  For any ``input'' $x$, and $w\in\Omega_A$, we write 
    $A(x;w)$ to mean running $A$ on $x$ with distinguished input $w$.
    We define the notation
    \[
        \Pr[A(x) = y] = \Pr[\{w\in\Omega_A \ : \ A(x;w)=y\}]
    \]
    to formalize ``the probability that randomized algorithm $A$ outputs $y$
    when given input $x$.''
\end{definition}
Note that $A(x;w)$ \emph{is a random variable}, as defined in the previous
section.  And since we like to suppress the placeholder variable $w$, we will
always just write $A(x)$.

Note that we can speak of running a randomized algorithm $A$ on a random
input. In this case, we'd usually take the sample space to include pairs
$(x,w)$, and think of $A(x;w)$ as a random variable.

\section{Probability Theory ``In Practice''}

Now that you've waded through that very quick review, I will close with
a discussion of how discrete probability is used, both in this class and
in other domains, from theoretical to applied.

As with many mathematical concepts, it is possible to maintain two modes of
thinking about probability theory: The first is intuitive, meaning that when
you read ``let $X$ be a uniformly random bit-string'', you don't have to
connect it a sample space, a measure, or a random variable. If you asked
``What's the probability that $X$ starts with a zero?'', you can say $1/2$
without the help of all this formalism. Similarly, you usually can answer
questions of independence intuitively.

This intuitive approach often proceeds without even mentioning a sample space
or measure. In a sense, statements like ``$\Pr[X=1]$'' use $\Pr$ as a symbol to
indicate that probability is being modeled, but do not mean to refer a
particular measure $\Pr$ as we've defined it. This is fine, but sometimes weird
steps happen: A sequence of steps in a proof will usually use the symbol $\Pr$
everywhere, \emph{even when referring multiple distinct probability measures on
different sample spaces}.  This is almost always fine, in that one can
formalize the true intention if necessary, and in fact it's best to not cloud
proofs with too much formalism.  Occasionally in this class we'll pause to
examine these statements, but the point will only be to understand, at a
mathematical level, what the symbols mean. The justification for the steps will
almost always be intuitive. We will see examples of this when learning
about perfect secrecy.

\end{document}

